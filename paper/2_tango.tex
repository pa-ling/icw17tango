% !TEX root = main.tex

\section{Google Tango}
Google Tango ist eine Plattform für Augmented Reality und Computer Vision für Android. Per Motion Tracking, Gyroskop und Beschleunigssensor ermittelt das Gerät seine Position im Raum. Über infrarotes Structured Light und Time-of-Flight-Messungen, sowie Stereo-Kameras werden Tiefenmessungen durchgeführt. Dadurch kann der Raum gescannt und in einer Punktwolke wiedergegeben werden. Diese kann dann z.B. dazu verwendet werden, virtuelle Objekte in den realen Raum zu platzieren oder die reale Welt virtuell abzubilden. Für all dies wird zusätzliche Hardware benötigt.\cite{fehling}\\
Ein besonderes Feature von Tango ist das "`Area Learning"'. Dabei wird ein "`Gedächtnis"' der Umgebung anhand von Landmarken aufgebaut. Verliert das Gerät die Orientierung findet es über das Area Learning wieder zurück.\cite{fehling}\\
In der Praxis dient es jedoch lediglich als Ergänzung zum Motion Tracking. Auf die gespeicherten Landmarken hat man keinen Zugriff, sodass eine Ortung über das Motion Tracking bzw. die TangoPointCloud erfolgen muss.