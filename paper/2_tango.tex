% !TEX root = main.tex

\section{Google Tango}
Google Tango ist eine Plattform für Augmented Reality und Computer Vision für Android. Per Motion Tracking, Gyroskop und Beschleunigssensor ermittelt das Gerät seine Position im Raum. Über infrarotes Structured Light und Time-of-Flight-Messungen, sowie Stereo-Kameras werden Tiefenmessungen durchgeführt. Dadurch kann der Raum gescannt und in einer Punktwolke wiedergegeben werden. Diese kann dann z.B. dazu verwendet werden, virtuelle Objekte im realen Raum zu platzieren oder die reale Welt virtuell abzubilden. Für all dies wird spezielle zusätzliche Hardware (z.B. IR-Projektor, Infrarotsensor) im Gerät benötigt.\cite{fehling}\\
Nachdem ich mich in meiner letzten Arbeit theoretisch mit den Konzepten von Google Tango auseinandergesetzt, hatte ich nun die Möglichkeit auch praktisch mit Google Tango zu arbeiten. Dazu nutzte ich das Lenovo Phab 2 Pro, das erste Tango-Gerät für Endverbraucher. Für den Einstieg bietet Google eine "`HowTos"' für Unity an. Folgendes wurde dabei umgesetzt:
\begin{enumerate}
	\item\textbf{Platzieren einer Kugel:} Nach dem Start der App wird die nächste beste Position zum Platzieren der Kugel gesucht und dort wird sie platziert. Anschließend kann man mit dem Gerät um diese herumlaufen.
	\item\textbf{Platzieren von Objekten bei Nutzereingabe:} Per Tap auf dem Bildschirm wird der Schnittpunkt zu auf dem berührten Pixel liegenden Oberfläche ermittelt und auf diesem wird ein Objekt platziert (hier: eine animierte Katze).
	\item\textbf{Scan eines Raums:} Es wird mit dem Gerät der Raum gescannt. Währenddessen wird ein Mesh des Raumes erstellt, welches als obj.Datei exportiert werden kann.
	\item\textbf{Visualisierung der Punktwolke:} Anstatt das normale Kamerabild oder eine fremde virtuelle Welt zu sehen, wird die reale Welt als Punktwolke, sowie sie vom Gerät "gesehen" wird dargestellt.
	\item\textbf{AreaLearning:} Bei der Applikation lassen sich Marken im Raum verteilen und speichern. Nach einem Neustart, werden diese Marken in etwa am gleichen Ort wieder platziert.
\end{enumerate}
Ein besonderes Feature von Tango ist das "`Area Learning"'. Dabei wird ein "`Gedächtnis"' der Umgebung anhand von Landmarken aufgebaut. Diese werden in einer Area Description File (ADF) Verliert das Gerät die Orientierung findet es über das Area Learning wieder zurück.\cite{fehling}\\
In der Praxis dient es jedoch lediglich als Ergänzung zum Motion Tracking. Auf die gespeicherten Landmarken hat man keinen Zugriff, sodass eine Ortung über das Motion Tracking bzw. die TangoPointCloud erfolgen muss.\\
Nach den Unity-"`HowTos"' konnte ich noch nicht sehr viel von Tango sehen, da viel von der Unity-Engine bzw. der dazugehörigen Tango-SDK abstrahiert wurde. Aus diesem Grund schaute ich mir auch die Java-API von Tango an. Zu dieser werden von Google keine Tutorials geliefert, jedoch haben sie auf Github eine Reihe von Beispiel-Applikationen, sowohl für Java (J) als auch für Unity (U):
\begin{itemize}
	\item\textbf{Hello Area Description / Area Description Management (J, U):} Erstellen von Speichern von ADFs. Die App zeigt zusätzlich wann man im ADF lokalisiert ist und wann nicht.
	\item\textbf{Area Learning (U):} Benutzen der Area Description Motion API und Platzieren von Objekten an benutzerspezifischen Stellen.
	\item\textbf{(Simple) Augmented Reality (J, U):} Platziert Mond und Erde an die nächst beste Position. Diese gibt es einmal als reine OpenGL ES Applikation und einmal mit der Rajawali Engine.
	\item\textbf{Find Floor (U):} Sucht in der aktuell gesehen Szene die niedrigste Ebene des Raumes (Boden).
	\item\textbf{Floor Planner (J):} Erstellt den Grundriss des gescannten Gebiets.
	\item\textbf{Green Screen (J):} Simuliert einen Greenscreen mithilfe der Tiefendaten (hintere Bereiche werden ausgeblendet).
	\item\textbf{Mesh Builder (J, U):} Gleiche Funktionsweise wie der Raumscan in Unity.
	\item\textbf{Model Correspondance (J):} Platziert ein Haus zwischen vier vom Nutzer gesetzten Punkten. Die größe des Hause hängt von der mit den Punkten markierten Fläche ab.
	\item\textbf{Motion Tracking (J, U)} Zeigt eine virtuelle Welt, in der man sich über Motion Tracking bewegen kann.
	\item\textbf{Occlusion (J):} Per Tap auf dem Bildschirm wird eine Erde platziert, welche von anderen realen Objekten verdeckt werden kann.
	\item\textbf{Point Cloud (J, U):} Visualisiert die Punktwolke. Der Nutzer kann zwischen verschiedenen Perspektiven wählen. Dabei handelt es sich um eine komplexere Variante als die von den Unity-"`HowTos"'.
	\item\textbf{Point To Point (J, U):} Der Nutzer setzt zwei Punkte der Tap. Die Strecke zwischen den beiden Punkten wird visualisiert und die Länge wird berechnet
\end{itemize}