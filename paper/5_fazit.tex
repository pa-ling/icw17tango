% !TEX root = main.tex

\section{Fazit}
Abschließend lässt sich sagen, dass Google Tango durch seine Sensoren ein paar mehr Features mitbringt. Das neue ARCore ist von außen sehr viel näher an ARKit angelehnt als Tango. Beide sind Versuche Augmented Reality einer breiteren Masse zur Verfügung zu stellen. Sofern dies erfolgreich verläuft und Endverbraucher die AR-Apps nutzen, wird wahrscheinlich auch Tango-ähnliche Technologie in neue Smartphones einfließen. So macht es Apple schon mit der Frontkamera des iPhones vor, welches Sensoren zu Tiefenmessung enthält. Die zusätzliche Hardware stellt für Apple-Produkte durch die geringe Produktvielfalt ein geringeres Problem dar als bei Android-Produkten, wo verschiedene Hersteller erst nachziehen müssen. Die Vergangenheit zeigt jedoch, dass eine populäre Technologie sich schnell durchsetzt, damit die Hersteller konkurrenzfähig bleiben. Die Frage ist jedoch, welchen Nutzen Endverbraucher aus Augmented Reality auf dem Smartphone ziehen können. ARCore und ARKit werden durch ihre geringen Hardware-Kosten genau diese Frage beantworten.