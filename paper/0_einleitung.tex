% !TEX root = main.tex

\section{Einleitung}

1968 entwickelte Ivan Sutherland das erste "`Head-Mounted Display"', welches simple Netzstrukturen anzeigen konnte. Dies gilt heute als Grundstein für Virtual (VR) und Augmented Reality (AR). 1992 wurde dann erstmals der Begriff "`augmented reality"' von Tom Caudell und David Mizell geprägt. Seitdem wuchs dieser Forschungsbereich stark an. Mehrere spezialisierte Konferenzen wurden gegründet und verschiedene Technologien wurden entwickelt \cite{ar_history}. Im Wesentlichen funktionieren diese entweder mit markerlosem oder markerbasiertem Tracking.\par
Im Folgenden soll ein Überblick über die Funktionsweise und Möglichkeiten von je einem Stellvertreter gegeben werden. Das ist PTC Vuforia für das markerbasierte Tracking und Google Tango für das markerlose.\par
Daraus wird sich ergeben, dass Google Tango technologisch aufgrund der größeren Datenmenge besser ist. Da Vuforia aber auf allen aktuellen mobilen Geräten mit einer Kamera funktioniert, hat es große Chancen auch weiterhin zu bestehen.