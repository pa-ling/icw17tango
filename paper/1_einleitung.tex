% !TEX root = main.tex

\section{Einleitung}

Google Tango wurde erstmals am 3. November 2014 der Öffentlichkeit zur Verfügung gestellt \cite{tango_release_notes}. Im Juni 2017 stellte Apple auf seiner Apple Worlwide Developers Conference iOS 11 mit ARKit vor \cite{ios11_announcement}. Als Antwort darauf ist das Ende August 2017 erschienende Google ARCore zu verstehen, welches im Gegensatz zu Tango ohne zusätzliche Hardware auskommt \cite{arcore_announcement}.\par
ARCore wird weitestgehend als Nachfolger von Google Tango angesehen \cite{cnet_arcore}\cite{heise_arcore}. Die offizielle Aussage hierzu ist: "`We’ve been developing the fundamental technologies that power mobile AR over the last three years with Tango, and ARCore is built on that work."'\cite{arcore_announcement} Ein weiteres Indiz für die Ablösung von Tango ist die Tatsache, dass es seit Juni keine neuen Releases der Plattform mehr gab, obwohl vorher immer mindestens monatlich eine neue Version veröffentlicht wurde \cite{tango_release_notes}. \par
Aus diesem Grund bin ich auch der Meinung, dass Tango ersetzt wurde und eher nach und nach Funktionalität aus der Tango SDK in ARCore umziehen werden, sofern die benötigte Hardware verfügbar gemacht wird.\par
In dieser Arbeit soll nun geklärt werden, welche Unterschiede zwischen Tango und dem ARCore bestehen, also welche Verluste man dadurch einbüßt, und wie konkurrenzfähig dies zum Apple ARKit ist.