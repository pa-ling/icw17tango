% !TEX root = main.tex

\section{Einleitung}

Google Tango wurde erstmals am 3. November 2014 der Öffentlichkeit zur Verfügung gestellt \cite{tango_release_notes}. Im Juni 2017 stellte Apple auf seiner Apple Worlwide Developers Conference iOS 11 mit ARKit vor \cite{ios11_announcement}. Als Antwort darauf ist das Ende August 2017 erschienene Google ARCore zu verstehen, welches im Gegensatz zu Tango ohne zusätzliche Hardware auskommt \cite{arcore_announcement}.

ARCore wird weitestgehend als Nachfolger von Google Tango angesehen \cite{cnet_arcore}\cite{heise_arcore}. Eine offizielle Aussage von Google hierzu ist: "`We’ve been developing the fundamental technologies that power mobile AR over the last three years with Tango, and ARCore is built on that work."'\cite{arcore_announcement} Dabei wird nicht direkt vom Ende der Tango-Plattform gesprochen. Ein weiteres starkes Indiz für die Ablösung von Tango ist jedoch die Tatsache, dass es seit Juni keine neuen Releases der Plattform mehr gab, obwohl vorher immer mindestens monatlich eine neue Version veröffentlicht wurde \cite{tango_release_notes}.

Aus diesem Grund bin ich auch der Meinung, dass Tango ersetzt wurde und eher nach und nach Funktionalität aus der Tango SDK in ARCore umziehen werden, sofern die benötigte Hardware verfügbar gemacht wird.

In dieser Arbeit soll nun geklärt werden, welche Unterschiede zwischen Tango und dem ARCore bestehen, also welche Verluste man dadurch einbüßt, und wie konkurrenzfähig dies zum Apple ARKit ist.

Die nötige Hardware für ARCore steht mir leider nicht zur Verfügung. Außerdem befindet sich ARCore noch im Preview-Status. Aufgrund der starken Ähnlichkeit von ARCore und ARKit sollte ein Vergleich zwischen Tango und ARKit hier jedoch ausreichen.