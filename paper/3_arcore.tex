% !TEX root = main.tex

\section{Google ARCore}
Google ARCore ist ebenfalls eine Augmented-Reality-Plattform. Offiziell ist es nur auf den Google Pixel Phones und dem Samsung Galaxy S8 unterstützt. Nach dem Verlassen des Preview-Status. Es basiert auf drei fundamentalen Konzepten \cite{arcore_overview}:

\subsection{Motion tracking} Wie bei Tango wird per Feature Points im gesehenen Bild die Position und Ausrichtung des Geräts im Raum ermittelt. Daten aus dem Gyroskop und Beschleunigungssensor (IMU) des Telefons werden hierbei ebenfalls mit den Bilddaten kombiniert. \cite{arcore_fundamentals}\par
Problem sind bei ruckartigen Bewegungen zu erwarten. Google Tango löste dieses Probleme mit dem Area Learning.

\subsection{Environmental understanding} Durch Analyse der Feature Points werden flache Oberflächen erkannt und können z.B. mit Objekten bestückt werden. \cite{arcore_fundamentals}\par
In Google Tango bietet z.B. die TangoPointCloud in der Unity-API per \texttt{findPlane}-Methode eine sehr ähnliche Funktionalität.

\subsection{Light estimation} Die reale Beleuchtung wird analysiert und ARCore stellt diese Informationen zur Verfügung, sodass virtuelle Objekte durch korrekte Beleuchtung realistischer aussehen. \cite{arcore_fundamentals}\par
Ein solches Feature gibt es in der Google Tango Plattform nicht.\\

ARCore baut laut Google auf der Arbeit von Tango auf. Der Vergleich in Tabelle \ref{arcore_vs_tango} zeigt, dass sich beide sehr weit voneinander entfernen, was zum einen daran liegt, das beide mit unterschiedlichen Daten arbeiten. Zum Anderen könnte dies aber auch am Apple ARKit liegen, was später noch deutlicher wird.

\begin{table}[h]
	\centering
	\begin{tabular}{|p{4cm}|p{4cm}|}
		\hline
		\textbf{ARCore} & \textbf{Tango}\\
		\hline
		Anchor & \\
		Config & TangoConfig\\
		Frame & TangoImageBuffer\\
		HitResult & \\
		LightEstimate & \\
		Plane & \\
		PlaneHitResult & \\
		PointCloud & TangoPointCloud\\
		PointCloudHitResult & \\
		Pose & TangoPoseData\\
		Session & Tango\\
		\hline
	\end{tabular}
	\caption{Gegenüberstellung der Schnittstellen von Google ARCore und Google Tango}
	\label{arcore_vs_tango}
\end{table}
