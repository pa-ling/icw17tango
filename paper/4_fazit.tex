% !TEX root = main.tex

\section{Fazit}
Wenn wir die beiden Technologien vergleichen ist es schwer zu beurteilen welche besser ist, beide haben ihre Vor- und Nachteile und Einsatzgebiete. Wenn man einen realen Raum in die Virtualität überführen möchte, ist Google Tango durch seine Umgebungswahrnehmung klar im Vorteil. Mit Vuforia hat man jedoch einen physischen Punkt an dem sich reale und virtuelle Welt treffen. Dieser kann vom Nutzer angefasst werden und könnte so entstehende Hürden bei der Nutzung verringern. Sofern der Nutzer z.B. für ein Spiel etwas greifen muss, ist diese Bindung an die reale Welt durchaus von Vorteil. Andererseits ist eine Vuforia-Anwendung ohne Marker in der Regel nutzlos.\par
Die weite Verbreitung von Vuforia ist jedoch nicht nur dem geschuldet, dass es markerlose Technologien für mobile Geräte noch nicht gab. Die Kosteneffizienz ist ebenfalls ein großer Faktor. Es braucht nicht einmal ein Highend-Smartphone, um eine Vuforia-App nutzen zu können. Außerdem gibt es Google Tango nur für Android und das wird sich vermutlich auch nicht so schnell ändern. Aus diesem Gründen könnte sich Tango nur langsam in der Verbreitung durchsetzen.\par
Auf der anderen Seite ist Android ebenfalls von Google, welches auf vielen verschieden Smartphones und Tablets läuft. Es ist also denkbar, dass z.B. in einer zukünftigen Generation der Pixel-Serie die Tango-Technologie verwendet wird und andere Hersteller dann dadurch nachziehen. Ohne diese Unterstützung wird sich diese Technologie wahrscheinlich nur schleichend durchsetzen.\par
Des Weiteren gab es schon lange keine neuen Technologien, die den Einsatzbereich für Smartphones und Tablets merklich erweitern. Der jüngste Vergleich hierzu wäre "`NFC"', welche viele Jahre vom Standard bis zur Integration in einer Vielzahl von Geräten benötigte \cite{nfc_iso}\cite{android_pay}.\par
Alles in allem ist Google Tango jedoch als mächtiger einzuschätzen, da einfach viel mehr Daten zur Verarbeitung zur Verfügung stehen. Man könnte viele Vuforia-Applikationen mit Tango wahrscheinlich nachbilden, indem man dem Raum nach einem Objekt/Marker scannt oder ihn sogar vollständig weglässt und stattdessen dahin projiziert, wo es sich am besten anbietet. Mit Vuforia einen Raum zu scannen oder Tiefe wahrzunehmen ist jedoch völlig unmöglich.