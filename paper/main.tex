
%% bare_conf.tex
%% V1.4b
%% 2015/08/26
%% by Michael Shell
%% See:
%% http://www.michaelshell.org/
%% for current contact information.
%%
%% This is a skeleton file demonstrating the use of IEEEtran.cls
%% (requires IEEEtran.cls version 1.8b or later) with an IEEE
%% conference paper.
%%
%% Support sites:
%% http://www.michaelshell.org/tex/ieeetran/
%% http://www.ctan.org/pkg/ieeetran
%% and
%% http://www.ieee.org/

%%*************************************************************************
%% Legal Notice:
%% This code is offered as-is without any warranty either expressed or
%% implied; without even the implied warranty of MERCHANTABILITY or
%% FITNESS FOR A PARTICULAR PURPOSE! 
%% User assumes all risk.
%% In no event shall the IEEE or any contributor to this code be liable for
%% any damages or losses, including, but not limited to, incidental,
%% consequential, or any other damages, resulting from the use or misuse
%% of any information contained here.
%%
%% All comments are the opinions of their respective authors and are not
%% necessarily endorsed by the IEEE.
%%
%% This work is distributed under the LaTeX Project Public License (LPPL)
%% ( http://www.latex-project.org/ ) version 1.3, and may be freely used,
%% distributed and modified. A copy of the LPPL, version 1.3, is included
%% in the base LaTeX documentation of all distributions of LaTeX released
%% 2003/12/01 or later.
%% Retain all contribution notices and credits.
%% ** Modified files should be clearly indicated as such, including  **
%% ** renaming them and changing author support contact information. **
%%*************************************************************************


% *** Authors should verify (and, if needed, correct) their LaTeX system  ***
% *** with the testflow diagnostic prior to trusting their LaTeX platform ***
% *** with production work. The IEEE's font choices and paper sizes can   ***
% *** trigger bugs that do not appear when using other class files.       ***                          ***
% The testflow support page is at:
% http://www.michaelshell.org/tex/testflow/



\documentclass[conference, ngerman]{IEEEtran}
% Some Computer Society conferences also require the compsoc mode option,
% but others use the standard conference format.
%
% If IEEEtran.cls has not been installed into the LaTeX system files,
% manually specify the path to it like:
% \documentclass[conference]{../sty/IEEEtran}

\usepackage[utf8]{inputenc}
\usepackage[ngerman]{babel}


% Some very useful LaTeX packages include:
% (uncomment the ones you want to load)


% *** MISC UTILITY PACKAGES ***
%
%\usepackage{ifpdf}
% Heiko Oberdiek's ifpdf.sty is very useful if you need conditional
% compilation based on whether the output is pdf or dvi.
% usage:
% \ifpdf
%   % pdf code
% \else
%   % dvi code
% \fi
% The latest version of ifpdf.sty can be obtained from:
% http://www.ctan.org/pkg/ifpdf
% Also, note that IEEEtran.cls V1.7 and later provides a builtin
% \ifCLASSINFOpdf conditional that works the same way.
% When switching from latex to pdflatex and vice-versa, the compiler may
% have to be run twice to clear warning/error messages.






% *** CITATION PACKAGES ***
%
%\usepackage{cite}
% cite.sty was written by Donald Arseneau
% V1.6 and later of IEEEtran pre-defines the format of the cite.sty package
% \cite{} output to follow that of the IEEE. Loading the cite package will
% result in citation numbers being automatically sorted and properly
% "compressed/ranged". e.g., [1], [9], [2], [7], [5], [6] without using
% cite.sty will become [1], [2], [5]--[7], [9] using cite.sty. cite.sty's
% \cite will automatically add leading space, if needed. Use cite.sty's
% noadjust option (cite.sty V3.8 and later) if you want to turn this off
% such as if a citation ever needs to be enclosed in parenthesis.
% cite.sty is already installed on most LaTeX systems. Be sure and use
% version 5.0 (2009-03-20) and later if using hyperref.sty.
% The latest version can be obtained at:
% http://www.ctan.org/pkg/cite
% The documentation is contained in the cite.sty file itself.






% *** GRAPHICS RELATED PACKAGES ***
%
\ifCLASSINFOpdf
  \usepackage[pdftex]{graphicx}
  % declare the path(s) where your graphic files are
  % \graphicspath{{../pdf/}{../jpeg/}}
  % and their extensions so you won't have to specify these with
  % every instance of \includegraphics
  % \DeclareGraphicsExtensions{.pdf,.jpeg,.png}
\else
  % or other class option (dvipsone, dvipdf, if not using dvips). graphicx
  % will default to the driver specified in the system graphics.cfg if no
  % driver is specified.
  \usepackage[dvips]{graphicx}
  % declare the path(s) where your graphic files are
  % \graphicspath{{../eps/}}
  % and their extensions so you won't have to specify these with
  % every instance of \includegraphics
  % \DeclareGraphicsExtensions{.eps}
\fi
% graphicx was written by David Carlisle and Sebastian Rahtz. It is
% required if you want graphics, photos, etc. graphicx.sty is already
% installed on most LaTeX systems. The latest version and documentation
% can be obtained at: 
% http://www.ctan.org/pkg/graphicx
% Another good source of documentation is "Using Imported Graphics in
% LaTeX2e" by Keith Reckdahl which can be found at:
% http://www.ctan.org/pkg/epslatex
%
% latex, and pdflatex in dvi mode, support graphics in encapsulated
% postscript (.eps) format. pdflatex in pdf mode supports graphics
% in .pdf, .jpeg, .png and .mps (metapost) formats. Users should ensure
% that all non-photo figures use a vector format (.eps, .pdf, .mps) and
% not a bitmapped formats (.jpeg, .png). The IEEE frowns on bitmapped formats
% which can result in "jaggedy"/blurry rendering of lines and letters as
% well as large increases in file sizes.
%
% You can find documentation about the pdfTeX application at:
% http://www.tug.org/applications/pdftex





% *** MATH PACKAGES ***
%
%\usepackage{amsmath}
% A popular package from the American Mathematical Society that provides
% many useful and powerful commands for dealing with mathematics.
%
% Note that the amsmath package sets \interdisplaylinepenalty to 10000
% thus preventing page breaks from occurring within multiline equations. Use:
%\interdisplaylinepenalty=2500
% after loading amsmath to restore such page breaks as IEEEtran.cls normally
% does. amsmath.sty is already installed on most LaTeX systems. The latest
% version and documentation can be obtained at:
% http://www.ctan.org/pkg/amsmath





% *** SPECIALIZED LIST PACKAGES ***
%
%\usepackage{algorithmic}
% algorithmic.sty was written by Peter Williams and Rogerio Brito.
% This package provides an algorithmic environment fo describing algorithms.
% You can use the algorithmic environment in-text or within a figure
% environment to provide for a floating algorithm. Do NOT use the algorithm
% floating environment provided by algorithm.sty (by the same authors) or
% algorithm2e.sty (by Christophe Fiorio) as the IEEE does not use dedicated
% algorithm float types and packages that provide these will not provide
% correct IEEE style captions. The latest version and documentation of
% algorithmic.sty can be obtained at:
% http://www.ctan.org/pkg/algorithms
% Also of interest may be the (relatively newer and more customizable)
% algorithmicx.sty package by Szasz Janos:
% http://www.ctan.org/pkg/algorithmicx




% *** ALIGNMENT PACKAGES ***
%
\usepackage{array}
\usepackage[justification=centering, font=small]{caption} 
% Frank Mittelbach's and David Carlisle's array.sty patches and improves
% the standard LaTeX2e array and tabular environments to provide better
% appearance and additional user controls. As the default LaTeX2e table
% generation code is lacking to the point of almost being broken with
% respect to the quality of the end results, all users are strongly
% advised to use an enhanced (at the very least that provided by array.sty)
% set of table tools. array.sty is already installed on most systems. The
% latest version and documentation can be obtained at:
% http://www.ctan.org/pkg/array



% *** SUBFIGURE PACKAGES ***
\ifCLASSOPTIONcompsoc
  \usepackage[caption=false,font=normalsize,labelfont=sf,textfont=sf]{subfig}
\else
  \usepackage[caption=false,font=footnotesize]{subfig}
\fi
% subfig.sty, written by Steven Douglas Cochran, is the modern replacement
% for subfigure.sty, the latter of which is no longer maintained and is
% incompatible with some LaTeX packages including fixltx2e. However,
% subfig.sty requires and automatically loads Axel Sommerfeldt's caption.sty
% which will override IEEEtran.cls' handling of captions and this will result
% in non-IEEE style figure/table captions. To prevent this problem, be sure
% and invoke subfig.sty's "caption=false" package option (available since
% subfig.sty version 1.3, 2005/06/28) as this is will preserve IEEEtran.cls
% handling of captions.
% Note that the Computer Society format requires a larger sans serif font
% than the serif footnote size font used in traditional IEEE formatting
% and thus the need to invoke different subfig.sty package options depending
% on whether compsoc mode has been enabled.
%
% The latest version and documentation of subfig.sty can be obtained at:
% http://www.ctan.org/pkg/subfig




% *** FLOAT PACKAGES ***
%
%\usepackage{fixltx2e}
% fixltx2e, the successor to the earlier fix2col.sty, was written by
% Frank Mittelbach and David Carlisle. This package corrects a few problems
% in the LaTeX2e kernel, the most notable of which is that in current
% LaTeX2e releases, the ordering of single and double column floats is not
% guaranteed to be preserved. Thus, an unpatched LaTeX2e can allow a
% single column figure to be placed prior to an earlier double column
% figure.
% Be aware that LaTeX2e kernels dated 2015 and later have fixltx2e.sty's
% corrections already built into the system in which case a warning will
% be issued if an attempt is made to load fixltx2e.sty as it is no longer
% needed.
% The latest version and documentation can be found at:
% http://www.ctan.org/pkg/fixltx2e


%\usepackage{stfloats}
% stfloats.sty was written by Sigitas Tolusis. This package gives LaTeX2e
% the ability to do double column floats at the bottom of the page as well
% as the top. (e.g., "\begin{figure*}[!b]" is not normally possible in
% LaTeX2e). It also provides a command:
%\fnbelowfloat
% to enable the placement of footnotes below bottom floats (the standard
% LaTeX2e kernel puts them above bottom floats). This is an invasive package
% which rewrites many portions of the LaTeX2e float routines. It may not work
% with other packages that modify the LaTeX2e float routines. The latest
% version and documentation can be obtained at:
% http://www.ctan.org/pkg/stfloats
% Do not use the stfloats baselinefloat ability as the IEEE does not allow
% \baselineskip to stretch. Authors submitting work to the IEEE should note
% that the IEEE rarely uses double column equations and that authors should try
% to avoid such use. Do not be tempted to use the cuted.sty or midfloat.sty
% packages (also by Sigitas Tolusis) as the IEEE does not format its papers in
% such ways.
% Do not attempt to use stfloats with fixltx2e as they are incompatible.
% Instead, use Morten Hogholm'a dblfloatfix which combines the features
% of both fixltx2e and stfloats:
%
% \usepackage{dblfloatfix}
% The latest version can be found at:
% http://www.ctan.org/pkg/dblfloatfix




% *** PDF, URL AND HYPERLINK PACKAGES ***
%
%\usepackage{url}
\usepackage[hidelinks]{hyperref}
% url.sty was written by Donald Arseneau. It provides better support for
% handling and breaking URLs. url.sty is already installed on most LaTeX
% systems. The latest version and documentation can be obtained at:
% http://www.ctan.org/pkg/url
% Basically, \url{my_url_here}.




% *** Do not adjust lengths that control margins, column widths, etc. ***
% *** Do not use packages that alter fonts (such as pslatex).         ***
% There should be no need to do such things with IEEEtran.cls V1.6 and later.
% (Unless specifically asked to do so by the journal or conference you plan
% to submit to, of course. )


\usepackage{gensymb}

% correct bad hyphenation here
\hyphenation{op-tical net-works semi-conduc-tor}


\begin{document}
\title{Google Tango vs Google ARCore vs Apple ARKit}
\author{\IEEEauthorblockN{Patrick Fehling}
\IEEEauthorblockA{Hochschule für Technik und Wirtschaft Berlin, Deutschland\\
E-Mail: p.fehling@student.htw-berlin.de\\
\today}}


\maketitle

\begin{abstract}
Google Tango war die erste markerlose "`Augmented Reality"'-Plattform für Smartphones. Das Apple ARKit und das ARCore aus dem eigenen Hause stellen große Konkurrenz dar, da sie im Gegensatz zu Tango keine extra Hardware benötigt. Am Beispiel von ARKit wird gezeigt was eine aktuelles Smartphone in AR leisten kann. Ein besonderer Fokus in dieser Arbeit liegt dabei auf dem Area Learning, welches noch ein Alleinstellungsmerkmal von Tango zu sein scheint.\\
\end{abstract}

\renewcommand\IEEEkeywordsname{Schlüsselbegriffe}
\begin{IEEEkeywords}
	Augmented Reality; Google Tango; Google ARCore, Apple ARKit.
\end{IEEEkeywords}

% !TEX root = main.tex

\section{Einleitung}

Google Tango wurde erstmals am 3. November 2014 der Öffentlichkeit zur Verfügung gestellt \cite{tango_release_notes}. Im Juni 2017 stellte Apple auf seiner Apple Worlwide Developers Conference iOS 11 mit ARKit vor \cite{ios11_announcement}. Als Antwort darauf ist das Ende August 2017 erschienende Google ARCore zu verstehen, welches im Gegensatz zu Tango ohne zusätzliche Hardware auskommt \cite{arcore_announcement}.\par
ARCore wird weitestgehend als Nachfolger von Google Tango angesehen \cite{cnet_arcore}\cite{heise_arcore}. Die offizielle Aussage hierzu ist: "`We’ve been developing the fundamental technologies that power mobile AR over the last three years with Tango, and ARCore is built on that work."' Ein weiteres Indiz für die Ablösung von Tango ist die Tatsache, dass es seit Juni keine neuen Releases der Plattform mehr gab, obwohl vorher immer mindestens monatlich eine neue Version veröffentlicht wurde \cite{tango_release_notes}. \par
Aus diesem Grund bin ich auch der Meinung, dass Tango ersetzt wurde und eher nach und nach Funktionalität aus der Tango SDK in ARCore umziehen werden, sofern die benötigte Hardware verfügbar gemacht wird.\par
In dieser Arbeit soll nun geklärt werden, welche Unterschiede zwischen Tango und dem ARCore bestehen, also welche Verluste man dadurch einbüßt, und wie konkurrenzfähig dies zum Apple ARKit ist.
% !TEX root = main.tex

\section{Google Tango}
Google Tango ist eine Plattform für Augmented Reality und Computer Vision für Android. Per Motion Tracking, Gyroskop und Beschleunigssensor ermittelt das Gerät seine Position im Raum. Über infrarotes Structured Light und Time-of-Flight-Messungen, sowie Stereo-Kameras werden Tiefenmessungen durchgeführt. Dadurch kann der Raum gescannt und in einer Punktwolke wiedergegeben werden. Diese kann dann z.B. dazu verwendet werden, virtuelle Objekte im realen Raum zu platzieren oder die reale Welt virtuell abzubilden. Für all dies wird spezielle zusätzliche Hardware (z.B. IR-Projektor, Infrarotsensor) im Gerät benötigt.\cite{fehling}\\
Nachdem ich mich in meiner letzten Arbeit theoretisch mit den Konzepten von Google Tango auseinandergesetzt, hatte ich nun die Möglichkeit auch praktisch mit Google Tango zu arbeiten. Dazu nutzte ich das Lenovo Phab 2 Pro, das erste Tango-Gerät für Endverbraucher. Für den Einstieg bietet Google eine "`HowTos"' für Unity an. Folgendes wurde dabei umgesetzt:
\begin{enumerate}
	\item\textbf{Platzieren einer Kugel:} Nach dem Start der App wird die nächste beste Position zum Platzieren der Kugel gesucht und dort wird sie platziert. Anschließend kann man mit dem Gerät um diese herumlaufen.
	\item\textbf{Platzieren von Objekten bei Nutzereingabe:} Per Tap auf dem Bildschirm wird der Schnittpunkt zu auf dem berührten Pixel liegenden Oberfläche ermittelt und auf diesem wird ein Objekt platziert (hier: eine animierte Katze).
	\item\textbf{Scan eines Raums:} Es wird mit dem Gerät der Raum gescannt. Währenddessen wird ein Mesh des Raumes erstellt, welches als obj.Datei exportiert werden kann.
	\item\textbf{Visualisierung der Punktwolke:} Anstatt das normale Kamerabild oder eine fremde virtuelle Welt zu sehen, wird die reale Welt als Punktwolke, sowie sie vom Gerät "gesehen" wird dargestellt.
	\item\textbf{AreaLearning:} Bei der Applikation lassen sich Marken im Raum verteilen und speichern. Nach einem Neustart, werden diese Marken in etwa am gleichen Ort wieder platziert.
\end{enumerate}
Ein besonderes Feature von Tango ist das "`Area Learning"'. Dabei wird ein "`Gedächtnis"' der Umgebung anhand von Landmarken aufgebaut. Diese werden in einer Area Description File (ADF) Verliert das Gerät die Orientierung findet es über das Area Learning wieder zurück.\cite{fehling}\\
In der Praxis dient es jedoch lediglich als Ergänzung zum Motion Tracking. Auf die gespeicherten Landmarken hat man keinen Zugriff, sodass eine Ortung über das Motion Tracking bzw. die TangoPointCloud erfolgen muss.\\
Nach den Unity-"`HowTos"' konnte ich noch nicht sehr viel von Tango sehen, da viel von der Unity-Engine bzw. der dazugehörigen Tango-SDK abstrahiert wurde. Aus diesem Grund schaute ich mir auch die Java-API von Tango an. Zu dieser werden von Google keine Tutorials geliefert, jedoch haben sie auf Github eine Reihe von Beispiel-Applikationen, sowohl für Java (J) als auch für Unity (U):
\begin{itemize}
	\item\textbf{Hello Area Description / Area Description Management (J, U):} Erstellen von Speichern von ADFs. Die App zeigt zusätzlich wann man im ADF lokalisiert ist und wann nicht.
	\item\textbf{Area Learning (U):} Benutzen der Area Description Motion API und Platzieren von Objekten an benutzerspezifischen Stellen.
	\item\textbf{(Simple) Augmented Reality (J, U):} Platziert Mond und Erde an die nächst beste Position. Diese gibt es einmal als reine OpenGL ES Applikation und einmal mit der Rajawali Engine.
	\item\textbf{Find Floor (U):} Sucht in der aktuell gesehen Szene die niedrigste Ebene des Raumes (Boden).
	\item\textbf{Floor Planner (J):} Erstellt den Grundriss des gescannten Gebiets.
	\item\textbf{Green Screen (J):} Simuliert einen Greenscreen mithilfe der Tiefendaten (hintere Bereiche werden ausgeblendet).
	\item\textbf{Mesh Builder (J, U):} Gleiche Funktionsweise wie der Raumscan in Unity.
	\item\textbf{Model Correspondance (J):} Platziert ein Haus zwischen vier vom Nutzer gesetzten Punkten. Die größe des Hause hängt von der mit den Punkten markierten Fläche ab.
	\item\textbf{Motion Tracking (J, U)} Zeigt eine virtuelle Welt, in der man sich über Motion Tracking bewegen kann.
	\item\textbf{Occlusion (J):} Per Tap auf dem Bildschirm wird eine Erde platziert, welche von anderen realen Objekten verdeckt werden kann.
	\item\textbf{Point Cloud (J, U):} Visualisiert die Punktwolke. Der Nutzer kann zwischen verschiedenen Perspektiven wählen. Dabei handelt es sich um eine komplexere Variante als die von den Unity-"`HowTos"'.
	\item\textbf{Point To Point (J, U):} Der Nutzer setzt zwei Punkte der Tap. Die Strecke zwischen den beiden Punkten wird visualisiert und die Länge wird berechnet
\end{itemize}
% !TEX root = main.tex

\section{Google ARCore}
Google ARCore ist ebenfalls eine Augmented-Reality-Plattform. Offiziell ist es nur auf den Google Pixel Phones und dem Samsung Galaxy S8 unterstützt. Nach dem Verlassen des Preview-Status sollen noch weitere Geräte folgen. Es basiert auf drei fundamentalen Konzepten \cite{arcore_overview}:
\begin{enumerate}
	\item \textbf{Motion tracking:} Wie bei Tango wird per Feature Points im gesehenen Bild die Position und Ausrichtung des Geräts im Raum ermittelt. Daten aus dem Gyroskop und Beschleunigungssensor (IMU) des Telefons werden hierbei ebenfalls mit den Bilddaten kombiniert. \cite{arcore_fundamentals}\par
	Problem sind bei ruckartigen Bewegungen zu erwarten. Google Tango löste dieses Probleme mit dem Area Learning.
	\item \textbf{Environmental understanding:}  Durch Analyse der Feature Points werden flache Oberflächen erkannt und können z.B. mit Objekten bestückt werden. \cite{arcore_fundamentals}\par
	In Google Tango bietet z.B. die TangoPointCloud in der Unity-API per \texttt{findPlane}-Methode eine sehr ähnliche Funktionalität.
	\item \textbf{Light estimation:} Die reale Beleuchtung wird analysiert und ARCore stellt diese Informationen zur Verfügung, sodass virtuelle Objekte durch korrekte Beleuchtung realistischer aussehen. \cite{arcore_fundamentals}\par
	Ein solches Feature gibt es in der Google Tango Plattform nicht.\\
\end{enumerate}

ARCore baut laut Google auf der Arbeit von Tango auf. Der Vergleich in Tabelle \ref{arcore_vs_tango} zeigt, dass sich beide sehr weit voneinander entfernen, was zum einen daran liegt, das beide mit unterschiedlichen Daten arbeiten. Zum Anderen könnte dies aber auch am Apple ARKit liegen, was später noch deutlicher wird.

\begin{table}[h]
	\centering
	\begin{tabular}{|p{4cm}|p{4cm}|}
		\hline
		\textbf{ARCore} & \textbf{Tango}\\
		\hline
		Anchor & \\
		Config & TangoConfig\\
		Frame & TangoImageBuffer\\
		HitResult & \\
		LightEstimate & \\
		Plane & \\
		PlaneHitResult & \\
		PointCloud & TangoPointCloud\\
		PointCloudHitResult & \\
		Pose & TangoPoseData\\
		Session & Tango\\
		\hline
	\end{tabular}
	\caption{Gegenüberstellung der Schnittstellen von Google ARCore und Google Tango}
	\label{arcore_vs_tango}
\end{table}

% !TEX root = main.tex

\section{Apple ARKit}
Das ARKit von Apple ist ein Augmented-Reality-Framework, welches mit iOS 11 veröffentlicht wurde \cite{apple_arkit}. ARKit und ARCore sind in ihrem Aufbau sehr ähnlich, sowohl in den Konzepten, als auch in den Schnittstellen. In Tabelle \ref{arkit_vs_arcore} ist dies verdeutlicht.

\begin{table}[h]
	\centering
	\begin{tabular}{|p{4cm}|p{4cm}|}
		\hline
		\textbf{ARKit} & \textbf{ARCore}\\
		\hline
		\multicolumn{2}{|c|}{\textbf{Konzepte}}\\
		\hline
		Visual Inertial Odometry & Motion Tracking\\
		Scene Understanding & Environmental Understanding\\
		Light Estimation & Light Estimation\\
		\hline
		\multicolumn{2}{|c|}{\textbf{Schnittstellen}}\\
		\hline
		ARAnchor & Anchor \\
		ARConfiguration & Config\\
		ARFrame & Frame\\
		ARHitTestResult & HitResult\\
		ARLightEstimate /
		\newline ARDirectionalLightEstimate & LightEstimate\newline\\
		ARPlaneAnchor & Plane\\
		ARSession & Session\\
		ARFaceAnchor & ---\\
		ARCamera & ---\\
		\hline
	\end{tabular}
	\caption{Gegenüberstellung von Apple ARKit und Google ARCore}
	\label{arkit_vs_arcore}
\end{table}

Womit das ARKit deutlich hervorsticht ist die "`Face-Based AR Experience"'. Mithilfe der "`TrueDepth Camera"' als Front-Kamera des iPhones kann die Position und das Aussehen des Gesichts erfasst werden und z.B. auf ein virtuelles Gesicht übertragen werden. Ein weiterer Punkt ist das komplette ausblenden des Hintergrunds.\cite{iphoneX}

Die Technologie dahinter ist der Tango-Hardware sehr ähnlich. Es gibt einen Infrarotsensor und einen "`Dot-Projector"', wodurch die Tiefe der Szene, also des Gesichts, ermittelt wird. \cite{iphoneX_display}

%eher fazit:
Ich denke es ist nur eine Frage der Zeit bis diese Technologie auch in der hinteren Kamera eingebaut wird. Die zusätzliche Hardware stellt für Apple-Produkte durch die geringe Produktvielfalt ein geringeres Problem dar als bei Android-Produkten, wo verschiedene Hersteller erst nachziehen müssen. Die Vergangenheit zeigt jedoch, dass eine populäre Technologie sich schnell durchsetzt, damit die Hersteller konkurrenzfähig bleiben. Die Frage ist jedoch, welchen Nutzen Endverbraucher aus Augmented Reality auf dem Smartphone ziehen können. ARCore und ARKit werden durch ihre geringen Kosten genau diese Frage beantworten.
% !TEX root = main.tex

\section{Fazit}
ARCore ist im Moment von außen deutlich näher am ARKit als an Google Tango.

%
Ich denke es ist nur eine Frage der Zeit bis diese Technologie auch in der hinteren Kamera eingebaut wird. Die zusätzliche Hardware stellt für Apple-Produkte durch die geringe Produktvielfalt ein geringeres Problem dar als bei Android-Produkten, wo verschiedene Hersteller erst nachziehen müssen. Die Vergangenheit zeigt jedoch, dass eine populäre Technologie sich schnell durchsetzt, damit die Hersteller konkurrenzfähig bleiben. Die Frage ist jedoch, welchen Nutzen Endverbraucher aus Augmented Reality auf dem Smartphone ziehen können. ARCore und ARKit werden durch ihre geringen Kosten genau diese Frage beantworten.

\pagestyle{plain}
\bibliographystyle{unsrt}
\bibliography{bibliography}

\end{document}


